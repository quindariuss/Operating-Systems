\documentclass{article}
\title{Midterm Exam: Operating Systems}
\author{Quin'darius Lyles-Woods}
\begin{document}
\maketitle

\section{How does the kernel know if an application is in an infinite loop?}
It can tell through the kernel and its interrupt driven hardware and software. It will catch it through an exception or trap.
\section{Linux file has three levels of security associated with it that matches the three classes of users that may access that file. What are those?}

\begin{itemize}
	\item Owner
	\begin{itemize}
		\item Read
		\item Write
		\item Execute
	\end{itemize}
	\item Group
	\begin{itemize}
		\item Read
		\item Write
		\item Execute
	\end{itemize}
	\item Other
	\begin{itemize}
		\item Read
		\item Write
		\item Execute
	\end{itemize}
\end{itemize}

\section{All computers follow roughly the same set of steps to transition from a power-off state to a running state. Can you enumerate the steps of the booting sequence?}
\begin{verbatim}
                           +----------------------+                                      
                           |                      |                                      
                           |                      |                                      
+----------------------+   |                      |                                      
|o o o                 |   |  Loading the Kernel  |           Running                    
+----------------------+   |     Into Memory      |------------Kernel-----+              
|                      |   |                      |                       |              
|                      |   |                      |                       |              
|    Power is Off      |   |                      |                       v              
|                      |   +----------------------+         +---------------------------+
|                      |               ^                    |                           |
|                      |               |                    |                           |
+----------------------+           Locates                  |   System Boot Complete    |
            |                       Kernel                  |                           |
            |                          |                    |                           |
            |                          |                    |                           |
            |                          |                    +---------------------------+
            |                          |                                                 
            |            +--------------------------+                                    
        Power On         |                          |                                    
            |            |                          |                                    
            |            |  Fixed Memory Location   |                                    
            |            |      That Contains:      |                                    
            +----------->|        Bootloader        |                                    
                         |           BIOS           |                                    
                         |                          |                                    
                         |                          |                                    
                         |                          |                                    
                         +--------------------------+                                    
\end{verbatim}
\pagebreak
\section{Output redirection in Linux/UNIX system. We use \(>\) as well as \(>>\) as the output direction. What is the semantic difference between the two option switches?}
You take the output of a program and write it to a file the manner in which is done is discussed below.
\begin{itemize}
	\item [\(>\)] This is used for writing to a file and if the file already exist it will overwrite it.
	\item [\(>>\)] This is used for writing to a file and if the file already exist it will append to the file.
\end{itemize}
\section{What does the command \texttt{chmod 765 foo} tell the computer to do?}
Chmod is a program that defines the permissions of files and directories. 
\begin{itemize}
	\item [7: Owner] Read Write and Execute
	\item [6: Group] Read and Write
	\item [5: Other] Read Only
\end{itemize}
So the command means that the owner has full access, group only read and write and the others group only read only.
\section{What do you mean by big-endian and little-endian system?}

\section{Keeping in mind the various definitions of \textbf{operating system}, consider whether the operating system should include applications such as web browsers and mail programs. Argue both that it should and that it should not, and support your answers.}

\section{Explain the difference between a batch processing and multiprogramming.}

\section{What is the purpose of the UNIX pipe command, i.e., vertical bar character \(|\)}

\section{Which of the following scheduling algorithms could result in starvation?}

\section{What do you know about BIOS? In the class we discussed that PC/BIOS PROM monitor is rather limited in its ability to access the system hardware. What new standard is being proposed to replace the traditional legacy BIOS?}

\section{What is the purpose of interrupts? What are the differences between a trap and an interrupt? What is the use of each function? How are multiple interrupts dealt with? Can traps be generated intentionally by a user program? If so, for what purpose?}

\section{What is the purpose of system calls, and how do system calls relate to the OS and to the concept of dual-mode (kernel mode and user mode) operation?}

\section{Why do some operating systems store the operating system in firmware, while others store it on disk?}

\section{Label the figure}

\section{Explain the figure}

\section{Explain different parts of the memory and what is the purpose of those memory.}

\section{Using Amdalh’s Law, calculate the speedup gain of an application that has 60 percent parallel component for (a) two processing cores and (b) four processing cores.}
\end{document}
